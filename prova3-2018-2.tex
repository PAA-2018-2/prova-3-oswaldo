\documentclass[12pt]{article}

\usepackage[utf8]{inputenc}
\usepackage[brazil]{babel}
\usepackage{amsmath,amssymb}
\usepackage{pdfsync}
\usepackage[all]{xy}
\usepackage{color}

\newcommand{\resposta}[1]{ \noindent {\bf Solução}.{\color{blue} #1}}

\title{{\large Universidade de Brasília \\ Instituto de Ciências Exatas \\
    Departamento de Ciência da Computação} \\[1cm]
  CIC 117536 - Projeto e Análise de Algoritmos \\[.5cm]  Terceira Prova \\[.5cm] Turma: B}
\author{{\bf NP-completude}}
\date{Prof. Flávio L. C. de Moura \\[.5cm] \today}

\begin{document}
\maketitle

\begin{enumerate}
\item {\bf (2.5 pontos)} O problema 2-SAT tem como instâncias as
  fórmulas lógicas formadas por conjunções de disjunções de até dois
  literais, onde um literal é uma variável booleana ou a negação de
  uma variável booleana. Por exemplo, a expressão a seguir é uma
  instância de 2-SAT:

  $$(x_1\lor \neg x_2)\land (\neg x_1 \lor \neg x_3) \land (x_1 \lor x_2) \land x_3$$

  Prove que 2-SAT $\in$ P.

  
  \resposta{
    Podemos notar que uma disjunção de apenas duas variáveis equivale a dizer
    que a negação de uma variável implica que a outra seja verdadeira, i.e.
    $(x_1 \lor x_2)\equiv (\neg x_1 \implies x_2)\land (\neg x_2 \implies x_1)$,
    o que pode ser facilmente percebido da tabela-verdade da disjunção.

    Desse modo, podemos reescrever o problema 2-SAT da Forma Normal Conjuntiva
    para a Formal Normal Implicativa, trocando as conjunções do problema pelas
    implicações equivalentes. Com isso, podemos notar que caso:
    $$x_1 \implies \neg x_1$$

    Precisamos que $x_1$ seja falso, visto que $x_1$ sendo
    verdadeiro criaria uma contradição na qual teriamos tanto $x_1$ quanto
    $\neg x_1$ sendo verdadeiros. Já no caso:
    $$\neg x_1 \implies x_1$$

    Precisamos que $\neg x_1$ seja falso, de modo análogo ao caso anterior.
    Assim, teremos problemas apenas se tivermos:
    $$(x_1 \implies \neg x_1) \land (\neg x_1 \implies x_1)$$

    Caso no qual precisaríamos de $x_1$ sendo falso e verdadeiro, o que
    significa que a expressão booleana não pode ser satisfeita.

    Se construirmos um grafo de implicações a partir da Forma Normal
    Implicativa do problema, notando que $(x_1 \implies x_2) \land
    (x_2 \implies x_3)$ significa que $(x_1 \implies x_3)$, podemos
    chegar a conclusão de que se $x_1$ e $\neg x_1$ existirem no mesmo
    componente fortemente conexo do grafo de implicações, a expressão
    booleana do problema em questão não pode ser satisfeita.

    Assim, usando um algoritmo ótimo para encontrar os componentes fortemente
    conexos, como o algoritmo de Kosaraju ou o algoritmo de Tarjan, podemos
    resolver o 2-SAT em $O(n+m)$, onde n é o número de váriaveis e m o número
    de cláusulas da Forma Normal Conjuntiva do problema. Logo, temos que
    2-SAT $\in$ P.
  }
  
\item {\bf (2.5 pontos)} Em aula, assumimos que SAT é um problema
  NP-completo (Teorema de Cook-Levin), e a partir deste fato mostramos
  que 3-SAT e CLIQUE também são problemas NP-completos. As reduções
  foram feitas de acordo com o seguinte diagrama:

  $$\xymatrix{
    SAT \ar[d] \\
    3\mbox{-}SAT \ar[d] \\
    CLIQUE 
  }$$
  
  Um ciclo Hamiltoniano é um ciclo simple que visita cada vértice de
  um grafo exatamente uma vez. Considere o problema de decisão
  HAM-CYCLE que pergunta se um dado grafo (não-dirigido) $G$ possui um
  ciclo Hamiltoniano. Mostre que HAM-CYCLE é um problema
  NP-completo. Sua solução deve ser construída a partir de SAT, 3-SAT
  ou CLIQUE. Caso, você não veja como reduzir diretamente HAM-CYCLE a
  partir destes, mas sabe como fazê-lo a partir de um certo problema
  $Q$ então inicialmente mostre que $Q$ é NP-completo a partir de SAT,
  3-SAT ou CLIQUE, e assim por diante. Digamos que você não saiba como
  mostrar que $Q$ é NP-completo diretamente a partir de SAT, 3-SAT ou
  CLIQUE, mas você sabe como fazê-lo a partir de outro problema $Q'$,
  e também sabe como mostrar que $Q'$ é NP-completo a partir de 3-SAT,
  por exemplo. Então o diagrama correspondente à sua solução seria:

  $$\xymatrix{
    SAT \ar[d] & & & \\
    3\mbox{-}SAT \ar[d]\ar[r] & Q' \ar[r] & Q \ar[r] & \mbox{HAM-CYCLE}  \\
    CLIQUE & & & 
  }$$

  E todas as reduções (de 3-SAT para $Q'$, de $Q'$ para $Q$ e de $Q$ para HAM-CYCLE) devem ser detalhadas na sua solução.

  \resposta{
    Primeiramente, podemos fácilmente notar que HAM-CYCLE é NP. Dado um
    caminho no grafo, basta percorre-lo e verificar se o vértice de partida
    é o mesmo do de destino e se, durante o percurso, algum vértice foi
    visitado mais de uma vez. Esse algoritmo de verificação pode ser
    executado em tempo polinominal e, com isso, HAM-CYCLE é NP.

    Agora, vamos mostrar que com uma fórmula 3-FNC podemos criar um
    grafo G que tem um Caminho Hamiltoniano se, e somente se, a fórmula
    puder ser satisfeita.

    Para isso, para cada váriavel da fórmula construímos um gadget de
    3k+1 vértices, onde k é o número de cláusula na fórmula. Nesse gadget
    um vértice acima, uma cadeia de vértices no centro e um vértice abaixo.
    O vértice de cima leva às duas pontas da cadeia do centro e às duas pontas
    da cadeia do centro levam ao vértice de baixo. Os vértice da cadeia, por
    sua vez levam ao anterior e ao próximo, de modo que seja possível percorrer
    o gadget da esquerda para direita e da direita para a esquerda.

    Assim, temos um caminho no qual a váriavel que o gadget representa
    é verdadeira e uma caminho no qual a váriavel que o gadget representa
    é falsa. Em um, partindo-se do vértice do topo, vai-se para a esquerda,
    percorre-se a cadeia da esquerda para a direita e desce-se para o
    vértice de baixo. No outro, partindo-se do vértice do topo, vai-se para a
    direita, percorre-se a cadeia da direita para a esquerda e desce-se para
    o vértice de baixo.

    Com isso, concatena-se o vértice de baixo de uma váriavel com o vértice de
    cima de outra de modo que no final so haja um grafo,
    no qual temos o vértice de topo
    de uma váriavel e o de baixo de outra. Além disso, cria-se vértices
    que representam as cláusulas, que devem ser ligados nos gadgets das
    variáveis de modo a representar a existência da váriavel ou da
    sua negação na cláusula.

    Assim, tem-se um grafo no qual para que exista um Caminho Hamiltoniano,
    deve ser possível satisfazer a fórmula 3-FNC que gerou esse grafo, visto
    que para se atravessar as cadeias de cada váriavel no grafo é necessário
    que a fórmula possa ser satisfeita.

    Portanto, temos que o Caminho Hamiltoniano é NP-Difícil. Com isso,
    podemos notar que, dado um grafo $G$ para qual temos o problema do
    Caminho Hamiltoniano, se adicionarmos um vértice e o ligarmos a todos
    os vértices de $G$, caímos em um problema do Ciclo Hamiltoniano. Logo,
    HAM-CYCLE é NP-Díficil e, assim, NP-Completo.
  }
  
\item {\bf (2.5 pontos)} Considere o seguinte jogo em um grafo
  (não-dirigido) $G$, que inicialmente contém 0 ou mais bolas de gude
  em seus vértices: um movimento deste jogo consiste em remover duas
  bolas de gude de um vértice $v\in G$, e adicionar uma bola a algum
  vértice adjacente de $v$. Agora, considere o seguinte problema: Dado
  um grafo $G$, e uma função $p(v)$ que retorna o número de bolas de
  gude no vértice $v$, existe uma sequência de movimentos que remove
  todas as bolas de $G$, exceto uma? Mostre que este problema é
  NP-completo. A mesma observação feita no exercício anterior vale
  aqui: a prova deve ser feita a partir de problemas que provamos
  serem NP-completos, e reduções intermediárias, caso existam, devem
  ser incluídas na solução.

  \resposta{
    Primeiro provamos que o problema é NP. Dado que uma sequência de
    movimentos seja executada, podemos verificar todos os vértices de G
    com a função $p(v)$ e saber se a sequência executada é uma solução.
    Sendo $p(v)$ executada em tempo polinomial, temos que esse
    algoritmo de verificação é executado em tempo polinomial e, logo,
    o problema é NP.

    Em seguida provamos que o problema é NP-Difícil. É fácil notar que
    ao realizarmos o movimento do jogo estamos fazendo o equivalente de
    tirar do jogo uma das bolas e mover a outra para um vértice adjacente.
    Assim, é simples de se perceber que se seguirmos no próximo movimento
    para o vértice para o qual movemos a bola anterior, podemos tirar
    mais uma bola do jogo e mover novamente a mesma bola. Note que a
    bola que está sendo movida jamais precisará ser retirada do jogo.

    Assim, podemos notar que dado um Ciclo Hamiltoniano no grafo, basta que
    se siga esse ciclo um número determinado de vezes que teremos uma solução
    do problema. Esse número pode ser determinado pela função $p(v)$, visto que
    a cada ciclo é retirada uma bola de todos os vértices, podemos fazer
    $p(v)$ em todos os vértices antes de começar a percorrer o ciclo
    e pegar o maior valor retornado como número de vezes a se percorrer
    o ciclo.

    Com isso, temos que esse problema pode ser reduzido ao HAM-CYCLE do
    exercício anterior, tornando-o NP-Difícil. Portanto, temos que o
    problema da questão é NP-Completo.
  }
  
\item {\bf (2.5 pontos)} Uma fórmula booleana em {\it forma normal conjuntiva com disjunção exclusiva (FNCX)} é uma conjunção de diversas cláusulas, e cada cláusula é uma disjunção exclusiva (XOR) de diversos literais. Lembre-se que a disjunção exclusiva é dada por:

  $$\begin{array}{|l|l|l|}\hline
  a & b & a \oplus b \\ \hline
  V & V & F \\ \hline
  V & F & V \\ \hline
  F & V & V \\ \hline
  F & F & F \\ \hline
\end{array}$$

  O problema FNCX-SAT pergunta se uma dada fórmula em FNCX é
  satisfatível. Mostre que o problema FNCX-SAT está em $P$, ou então
  que FNCX-SAT é NP-completo. No último caso, a mesma observação feita
  nos dois exercícios anteriores vale aqui: a prova deve ser feita a
  partir de problemas que provamos serem NP-completos, e reduções
  intermediárias, caso existam, devem ser incluídas na solução.

  \resposta{
    A disjunção exclusiva pode ser vista como uma adição em módulo de dois.
    Assim, dada uma fórmula em FNCX, podemos transformá-la em um sistema
    linear de equações, e.g. $(x_1 \oplus \neg x_2) \land (x_1 \oplus x_3)
    \land (x_2 \oplus x_3)$ pode ser transformado em:
    $$x_1 + x_2 = 0 \mod 2$$
    $$x_1 + x_3 = 1 \mod 2$$
    $$x_2 + x_3 = 1 \mod 2$$

    Visto que $\neg x = (x \oplus 1)$. Assim, podemos resolver o sistema
    de equações em tempo cúbico por Eliminação Gaussiana e, com isso,
    temos FNCX-SAT $\in$ P.
  }
\end{enumerate}

\end{document}

%%% Local Variables:
%%% mode: latex
%%% TeX-master: t
%%% End:
